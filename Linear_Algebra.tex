\documentclass{report}
% Comment the following line to NOT allow the usage of umlauts
\usepackage[utf8]{inputenc}
% Uncomment the following line to allow the usage of graphics (.png, .jpg)
\pagestyle{headings}
\usepackage{geometry}
\geometry{left=3cm,right=3cm,top=3cm,bottom=3cm}

\usepackage[usenames,dvipsnames]{color}
\usepackage[colorlinks,linkcolor=NavyBlue,anchorcolor=red,citecolor=green]{hyperref}
\usepackage{graphicx}
\usepackage{enumerate}
\usepackage{multirow}
\usepackage{float}
\usepackage{extarrows}
\usepackage{array,makecell}

\usepackage{mathtools}
\usepackage{bm}
\usepackage[table,xcdraw]{xcolor}
\usepackage{amsmath,amsfonts,amssymb,mathrsfs}
\usepackage{tikz,tikz-cd}
\usetikzlibrary{arrows.meta,arrows,graphs,graphs.standard,calc,decorations.markings}
\usepackage{arydshln}

\usepackage[thmmarks,amsmath]{ntheorem}
\theorembodyfont{\upshape}
\newtheorem{definition}{Definition}[section]
\newtheorem{example}{Example}[section]
\newtheorem{proposition}{Proposition}[section]
\newtheorem{corollary}{Corollary}[section]
\newtheorem{theorem}{Theorem}[section]
\newtheorem{lemma}{Lemma}[section]
\theoremstyle{nonumberplain}
\theoremheaderfont{\itshape}
\theorembodyfont{\normalfont}
\theoremsymbol{\\ \rightline{$\square$}}
\newtheorem{proof}{Proof.}

\newcommand{\pluseq}{\mathrel{{+}{=}}}
\newcommand{\minuseq}{\mathrel{{-}{=}}}
\newcommand\scalemath[2]{\scalebox{#1}{\mbox{\ensuremath{\displaystyle #2}}}}
% Start the document
\begin{document}


~\\
\begin{center}
	~\\
	\vspace{6em}
	\textsc{\Huge Linear Algebra}
	~\\
	\vspace{2.5em}
	{\Large }
	~\\
	\vspace{6em}
	\textsf{H.C.}
	~\\
	\vspace{5in}
	{\large Latest Update: \today}
\end{center}
\tableofcontents
% Create a new 1st level heading
\chapter{Determinant}
\section{Calculation Techniques}
\begin{example}\label{example:Upper triangularization}
	\begin{align*}
		D_{n}=\left|\begin{array}{cccc}
			x      & a      & \cdots & a      \\
			a      & x      & \cdots & a      \\
			\vdots & \vdots & \ddots & \vdots \\
			a      & a      & \cdots & x
		\end{array}\right|=\left(x+(n-1)a\right)\left(x-a\right)^{n-1}.
	\end{align*}
\end{example}
\begin{proof}
	\textbf{Upper triangularization}. There are two possible ways to achieve this.
	\begin{enumerate}
		\item Multiple subtraction. Subtract row 1 from row 2, row 3, ... , row $n$ to get a shape like
		      \begin{equation*}
			      \scalemath{0.55}{
				      \left|\begin{array}{ccccc}
					      *      & * & * & \cdots & * \\
					      *      & * &   &        &   \\
					      *      &   & * &        &   \\
					      \vdots &   &   & \ddots &   \\
					      *      &   &   &        & *
				      \end{array}\right|}
		      \end{equation*}
		      and then eliminate the first column.
		      \begin{equation*}
			      \begin{aligned}
				      D_{n}=\left|\begin{array}{cccc}
					      x      & a      & \cdots & a      \\
					      a      & x      & \cdots & a      \\
					      \vdots & \vdots & \ddots & \vdots \\
					      a      & a      & \cdots & x
				      \end{array}\right| & \xlongequal{\scalemath{0.65}{\arraycolsep=10pt\def\arraystretch{0.85}
						      \begin{array}{l}
							      \mathtt{r_2\minuseq r_1} \\
							      \cdots                   \\
							      \mathtt{r_n\minuseq r_1}
						      \end{array}}}
				      \left|\begin{array}{cccc}
					      x      & a   & \cdots & a   \\
					      a-x    & x-a &        &     \\
					      \vdots &     & \ddots &     \\
					      a-x    &     &        & x-a
				      \end{array}\right|                                                                                                \\
				                                                   & \xlongequal{\mathtt{c_1\pluseq c_2+\cdots+c_n}}\left|\begin{array}{cccc}
					      x+(n-1)a & a   & \cdots & a   \\
					               & x-a &        &     \\
					               &     & \ddots &     \\
					               &     &        & x-a
				      \end{array}\right|.
			      \end{aligned}
		      \end{equation*}
		\item Accumulation. Add row 2, row 3, ... , row $n$ to row 1.
		      \begin{equation*}
			      \begin{aligned}
				      D_{n}=\left|\begin{array}{cccc}
					      x      & a      & \cdots & a      \\
					      a      & x      & \cdots & a      \\
					      \vdots & \vdots & \ddots & \vdots \\
					      a      & a      & \cdots & x
				      \end{array}\right| & \xlongequal{\mathtt{r_1\pluseq r_2+\cdots+r_n}}
				      \left|\begin{array}{cccc}
					      x+(n-1)a & x+(n-1)a & \cdots & x+(n-1)a \\
					      a        & x        & \cdots & a        \\
					      \vdots   & \vdots   & \ddots & \vdots   \\
					      a        & a        & \cdots & x
				      \end{array}\right|                                                                                \\
				                                                   & =\left(x+(n-1)a\right)\left|\begin{array}{cccc}
					      1      & 1      & \cdots & 1      \\
					      a      & x      & \cdots & a      \\
					      \vdots & \vdots & \ddots & \vdots \\
					      a      & a      & \cdots & x
				      \end{array}\right|          \\
				                                                   & \xlongequal{\scalemath{0.65}{\arraycolsep=1.4pt\def\arraystretch{0.85}
						      \begin{array}{l}
							      \mathtt{r_2\minuseq \mathit{a}* r_1} \\
							      \cdots                               \\
							      \mathtt{r_n\minuseq \mathit{a}* r_1}
						      \end{array}}}
				      \left(x+(n-1)a\right)\left|
				      \begin{array}{cccc}
					      1 & 1   & \cdots & 1   \\
					        & x-a &        &     \\
					        &     & \ddots &     \\
					        &     &        & x-a
				      \end{array}\right|.
			      \end{aligned}
		      \end{equation*}
	\end{enumerate}
\end{proof}
\begin{example} Suppose $x_i\ne a_i\;(i=1,2,\cdots,n)$.
	\begin{equation*}
		\begin{aligned}
			D_{n}=\left|\begin{array}{cccc}
				x_{1}  & a_{2}  & \cdots & a_{n}  \\
				a_{1}  & x_{2}  & \cdots & a_{n}  \\
				\vdots & \vdots & \ddots & \vdots \\
				a_{1}  & a_{2}  & \cdots & x_{n}
			\end{array}\right|=\left(1+\sum_{i=1}^{n} \frac{a_{i}}{x_{i}-a_{i}}\right) \prod_{i=1}^{n}\left(x_{i}-a_{i}\right).
		\end{aligned}
	\end{equation*}
\end{example}
\begin{proof}
	\textbf{Increase order}. Consider
	\begin{equation*}
		\begin{aligned}
			D_{n}=\left|\begin{array}{cccc}
				x_{1}  & a_{2}  & \cdots & a_{n}  \\
				a_{1}  & x_{2}  & \cdots & a_{n}  \\
				\vdots & \vdots & \ddots & \vdots \\
				a_{1}  & a_{2}  & \cdots & x_{n}
			\end{array}\right|=
			\left|\begin{array}{c;{2pt/2pt}ccc}
				1           & a_{1} & \cdots & a_{n} \\\hdashline[2pt/2pt]
				            &       &        &       \\
				0           &       & D_n    &       \\
				\phantom{a} &       &        &
			\end{array}\right|.
		\end{aligned}
	\end{equation*}
	Then we can use the multiple subtraction method in Example~\ref{example:Upper triangularization} to get an upper triangular determinant.
	\begin{equation*}
		\begin{aligned}
			D_{n}= & \left|\begin{array}{ccccc}
				1      & a_{1}     & a_2     & \cdots & a_{n}   \\
				-1     & x_1-a_{1} &         &        &         \\
				-1     &           & x_2-a_2 &        &         \\
				\vdots &           &         & \ddots &         \\
				-1     &           &         &        & x_n-a_n
			\end{array}\right|  \\
			=      & \left|\begin{array}{ccccc}
				1+\sum_{i=1}^{n} \frac{a_{i}}{x_{i}-a_{i}} & a_{1}     & a_2     & \cdots & a_{n}   \\
				                                           & x_1-a_{1} &         &        &         \\
				                                           &           & x_2-a_2 &        &         \\
				                                           &           &         & \ddots &         \\
				                                           &           &         &        & x_n-a_n
			\end{array}\right|.
		\end{aligned}
	\end{equation*}
\end{proof}

\begin{example}Suppose $a\ne b$.
	\begin{align*}
		D_{n}=\left|\begin{array}{cccc}
			x      & a      & \cdots & a      \\
			b      & x      & \cdots & a      \\
			\vdots & \vdots & \ddots & \vdots \\
			b      & b      & \cdots & x
		\end{array}\right|=\frac{a(x-b)^{n}-b(x-a)^{n}}{a-b}.
	\end{align*}
\end{example}
\begin{proof}
	\textbf{Recurrence}. By splitting the first column we get
	\begin{equation*}
		\begin{aligned}
			D_{n} & =\frac{a}{b}\left|\begin{array}{cccc}
				b      & b      & \cdots & b      \\
				b      & x      & \cdots & a      \\
				\vdots & \vdots & \ddots & \vdots \\
				b      & b      & \cdots & x
			\end{array}\right|+\left|\begin{array}{cccc}
				x-a & a      & \cdots & a      \\
				0   & x      & \cdots & a      \\
				0   & \vdots & \ddots & \vdots \\
				0   & b      & \cdots & x
			\end{array}\right| \\
			      & =\frac{a}{b}\left|\begin{array}{cccc}
				b & b   & \cdots & b      \\
				  & x-b & \cdots & a-b    \\
				  &     & \ddots & \vdots \\
				  &     &        & x-b
			\end{array}\right|+
			\left|\begin{array}{c;{2pt/2pt}ccc}
				x-a         & a & \cdots  & a \\\hdashline[2pt/2pt]
				            &   &         &   \\
				0           &   & D_{n-1} &   \\
				\phantom{a} &   &         &
			\end{array}\right|                                                             \\
			      & =a(x-b)^{n-1}+(x-a) D_{n-1} .
		\end{aligned}
	\end{equation*}
	Transpose it to obtain another recurrence relation
	\begin{equation*}
		\begin{aligned}
			D_{n}=D_{n}^{T}=b(x-a)^{n-1}+(x-b) D_{n-1}.
		\end{aligned}
	\end{equation*}
	Solve $D_n$ from the two recurrence equations.
\end{proof}
\newpage
\chapter{Matrix}
\section{Properties}
\begin{proposition}[properties of square matrices] Let $A, B$ be $n\times n$ matrices.
	\begin{enumerate}
		\item If $A,B$ is invertible,
		      \[
			      (AB)^{-1}=B^{-1}A^{-1},\quad(kA)^{-1}=\frac{1}{k}A^{-1}(k\ne 0),\quad \left|A^{n}\right|=\left|A\right|^n(n\in\mathbb{Z}).
		      \]
		\item
		      \[
			      (AB)^{\mathrm{T}}=B^{\mathrm{T}}A^{\mathrm{T}},\quad(kA)^{\mathrm{T}}=kA^{\mathrm{T}},\quad\left|A^{\mathrm{T}}\right|=\left|A\right|, (A+B)^{\mathrm{T}}=A^{\mathrm{T}}+B^{\mathrm{T}}.
		      \]
		\item
		      \[
			      \mathrm{adj}(AB)=\mathrm{adj}(B)\mathrm{adj}(A),\quad\mathrm{adj}(kA)=k^{n-1}\mathrm{adj}(A),\quad\left|\mathrm{adj}(A)\right|=\left|A\right|^{n-1},\quad\mathrm{adj}(\mathrm{adj}(A))=\left|A\right|^{n-2}A.
		      \]
		\item The mappings ${}^{-1}$, ${}^{\mathrm{T}}$, $\mathrm{adj}$ are commutable.
	\end{enumerate}
\end{proposition}
\begin{proposition}[properties of block matrices]
	Suppose $A=(A_{ij})$.
	\begin{enumerate}
		\item
		      \[
			      A^{\mathrm{T}}=(A_{ji}^{\mathrm{T}}).
		      \]
		\item If $A,B$ is invertible,
		      \[
			      \begin{pmatrix}
				      0 & A \\
				      B & 0
			      \end{pmatrix}^{-1}
			      =
			      \begin{pmatrix}
				      0      & B^{-1} \\
				      A^{-1} & 0
			      \end{pmatrix}
		      \]
	\end{enumerate}
\end{proposition}
\begin{proposition}[properties of block diagonal matrices]
	Suppose $A=\mathrm{diag}(A_1,A_2,\cdots,A_n)$, $A_i(i=1,2,\cdots,n)$ are square matrices.
	\begin{enumerate}
		\item
		      \[
			      \mathrm{diag}(A_1,A_2,\cdots,A_n)^k=\mathrm{diag}(A_1^k,A_2^k,\cdots,A_n^k)\quad(k\in\mathbb{Z}_{\ge0}).
		      \]
		      Furthermore, if $A_i(i=1,2,\cdots,n)$ are invertible,
		      \[
			      \mathrm{diag}(A_1,A_2,\cdots,A_n)^k=\mathrm{diag}(A_1^k,A_2^k,\cdots,A_n^k)\quad(k\in\mathbb{Z}).
		      \]
		\item
		      \[
			      \left|\mathrm{diag}(A_1,A_2,\cdots,A_n)\right|=\left|A_1\right|\left|A_n\right|\cdots\left|A_n\right|.
		      \]
	\end{enumerate}
\end{proposition}



\section{Linear Equations}
\subsection{Homogeneous Linear Equations}
Let $A_{m\times n}=(\alpha_1,\alpha_2,\cdots,\alpha_n)\in M_{m,n}(F)$. There are two possible solution sets for the equation $Ax=0$.
\begin{enumerate}
	\item $Ax=0$ has unique solution $x=0$.
	\item $Ax=0$ has infinitely many solutions.
\end{enumerate}

\noindent\textbf{Rank}
\begin{enumerate}
	\item $Ax=0$ has unique solution $x=0\iff r(A)=n$.
	\item $Ax=0$ has infinitely many solutions $\iff r(A)<n$.
\end{enumerate}

\noindent\textbf{Row Echelon Form}\\~
Suppose $A$ is equivalent to the following row echelon form
\[
	\left(\begin{array}{ccccccccccc}
			1                              & *                                     & \cdots & *      & * & \cdots & *      & *      & *      & \cdots & *      \\\cdashline{1-3}[3pt/3pt]
			\multicolumn{3}{c;{3pt/3pt}}{} & 1                                     & *      & \cdots & * & *      & *      & \cdots & *                        \\\cdashline{4-6}[3pt/3pt]
			                               &                                       &        &        &   &        & \ddots & \vdots & \vdots & \ddots & \vdots \\
			\multicolumn{7}{c;{3pt/3pt}}{} & 1                                     & *      & \cdots & *                                                       \\\hdashline[3pt/3pt]
			                               & \multicolumn{9}{c}{0_{(m-r)\times n}} &
		\end{array}\right).
\]
$r$ denotes the number of the nonzero rows.
\begin{enumerate}
	\item $Ax=0$ has unique solution $x=0\iff r=n$.
	\item $Ax=0$ has infinitely many solutions $\iff r<n$.
\end{enumerate}

\noindent\textbf{Spanned Space of Column Vectors}\\~
If we see $A$ as $(\alpha_1,\alpha_2,\cdots,\alpha_n)$, where $\alpha_j\in F^m\;(j=1,2\cdots,n)$, then
\[
	Ax=0\iff x_1\alpha_1+x_2\alpha_2+\cdots+x_n\alpha_n=0.
\]
\begin{enumerate}
	\item $Ax=0$ has unique solution $x=0\iff \alpha_1,\alpha_2,\cdots,\alpha_n$ are linearly independent $\iff\dim\operatorname{span}(\alpha_1,\alpha_2,\cdots,\alpha_n)=n$.
	\item $Ax=0$ has infinitely many solutions $x=0\iff \alpha_1,\alpha_2,\cdots,\alpha_n$ are linearly dependent $\iff\dim\operatorname{span}(\alpha_1,\alpha_2,\cdots,\alpha_n)<n$.
\end{enumerate}

\noindent\textbf{Linear Mapping}\\~
If we see $A$ as a linear mapping $\mathcal{A}\in\operatorname{Hom}_{\mathsf{Vect}_F}(F^n,F^m)$, that is
\begin{align*}
	\mathcal{A}:F^n&\longrightarrow F^m\\
	x&\longmapsto Ax,
\end{align*}
then the solution space of $Ax=0$ is $\ker\mathcal{A}$. Note
\[
y\in \operatorname{im}\mathcal{A}\iff \exists x\in F^n,y=Ax\iff \exists x\in F^n,y=x_1\alpha_1+x_2\alpha_2+\cdots+x_n\alpha_n\iff y\in\operatorname{span}(\alpha_1,\alpha_2,\cdots,\alpha_n).
\]
We have $\operatorname{im}\mathcal{A}=\operatorname{span}(\alpha_1,\alpha_2,\cdots,\alpha_n)$, which implies
\[
	r(A)=\dim\operatorname{span}(\alpha_1,\alpha_2,\cdots,\alpha_n)=\dim\operatorname{im}\mathcal{A}.
\]
\begin{enumerate}
	\item $Ax=0$ has unique solution $x=0\iff\mathcal{A}\text{ is injective}\iff\ker\mathcal{A}=0\iff \operatorname{im}\mathcal{A}=F^m$.
	\item $Ax=0$ has infinitely many solutions $x=0\iff\mathcal{A}\text{ is not injective}\iff\dim\ker\mathcal{A}>0\iff \dim\operatorname{im}\mathcal{A}<n$.
\end{enumerate}

\noindent\textbf{Orthogonal Complement}\\~
If we see $A$ as $(\beta_1,\beta_2,\cdots,\beta_n)^T$, where $\beta_i\in F^n\;(i=1,2\cdots,m)$, then 
\[
	Ax=0\iff \left\langle\beta_i,x\right\rangle=\beta_i^Tx=0,\;\left(i=1,2\cdots,m\right).
\]
The solution space of $Ax=0$ is
\[
	\bigcap_{i=1}^m\operatorname{span}\left(\beta_i^T\right)^{\perp}.\]

\noindent\textbf{Determinant}\\~
If $A$ is a square matrix, then
\begin{enumerate}
	\item $Ax=0$ has unique solution $x=0\iff \left|A\right|\ne 0$.
	\item $Ax=0$ has infinitely many solutions $\iff \left|A\right|=0$.
\end{enumerate}

\subsection{Nonhomogeneous Linear Equations}
Let $A_{m\times n}=(\alpha_1,\alpha_2,\cdots,\alpha_n)$. There are two possible results for the equation $Ax=b$:
\begin{enumerate}
	\item $Ax=b$ has no solution.
	\item $Ax=b$ has at least one solution.
	\begin{enumerate}
		\item $Ax=b$ has a unique solution.
		\item $Ax=b$ has infinitely many solutions.
	\end{enumerate}
\end{enumerate}

\noindent\textbf{Rank}
\begin{enumerate}
	\item $Ax=b$ has no solution $\iff r\left(A\mid b\right)=r(A)+1$.
	\item $Ax=b$ has at least one solution $\iff r\left(A\mid b\right)=r(A)$.
	\begin{enumerate}
		\item $Ax=b$ has a unique solution $\iff r(A)=n$.
		\item $Ax=b$ has infinitely many solutions $\iff r(A)<n$..
	\end{enumerate}
\end{enumerate}

\noindent\textbf{Row Echelon Form}\\~
Suppose $\left(A\mid b\right)$ is equivalent to the following row echelon form
\[
	\left(\begin{array}{ccccccccccc|c}
			1                              & *                                     & \cdots & *                 & * & \cdots & *      & *      & *      & \cdots & *      & d_1     \\\cdashline{1-3}[3pt/3pt]
			\multicolumn{3}{c;{3pt/3pt}}{} & 1                                     & *      & \cdots            & * & *      & *      & \cdots & *      & d_2                        \\\cdashline{4-6}[3pt/3pt]
			                               &                                       &        &                   &   &        & \ddots & \vdots & \vdots & \ddots & \vdots & \vdots \\
			\multicolumn{7}{c;{3pt/3pt}}{} & 1                                     & *      & \cdots            & * & d_r                                                           \\\hdashline[3pt/3pt]
			& \multicolumn{9}{c}{0_{1\times n}} &        & d_{r+1}\\
			                               & \multicolumn{9}{c}{0_{(m-r-1)\times n}} &        & 0_{(m-r-1)\times 1}
		\end{array}\right)
\]
$r$ denotes the number of the nonzero rows.
\begin{enumerate}
	\item $Ax=b$ has no solution $\iff d_{r+1}\ne 0$.
	\item $Ax=b$ has at least one solution $\iff d_{r+1}=0$..
	\begin{enumerate}
		\item $Ax=b$ has a unique solution $\iff r=n$.
		\item $Ax=b$ has infinitely many solutions $\iff r<n$..
	\end{enumerate}
\end{enumerate}

\noindent\textbf{Linear Mapping}\\~
If we see $A$ as a linear mapping $\mathcal{A}\in\operatorname{Hom}_{\mathsf{Vect}_F}(F^n,F^m)$, that is
\begin{align*}
	\mathcal{A}:F^n&\longrightarrow F^m\\
	x&\longmapsto Ax,
\end{align*}
then the solution space of $Ax=b$ is $\mathcal{A}^{-1}(b)+\ker\mathcal{A}$.
\begin{enumerate}
	\item $Ax=b$ has no solution $\iff b\notin \operatorname{im}\mathcal{A}$.
	\item $Ax=b$ has at least one solution $\iff b\in \operatorname{im}\mathcal{A}$.
	\begin{enumerate}
		\item $Ax=b$ has a unique solution $\iff\mathcal{A}\text{ is injective}\iff\ker\mathcal{A}=0\iff \operatorname{im}\mathcal{A}=F^m$.
		\item $Ax=b$ has infinitely many solutions $x=0\iff\mathcal{A}\text{ is not injective}\iff\dim\ker\mathcal{A}>0\iff \dim\operatorname{im}\mathcal{A}<n$.
	\end{enumerate}
\end{enumerate}

\noindent\textbf{Determinant}\\~
If $A$ is a square matrix, then

\begin{enumerate}
	\item $Ax=b$ has unique solution $\iff \left|A\right|\ne 0$.
	\item $Ax=b$ has no  solution or have infinitely many solutions $\iff \left|A\right|=0$.
\end{enumerate}

\chapter*{Appendix}



\section*{1.}


\newpage

%\begin{thebibliography}{99}  
%	\bibitem{ref1}Gall J F L. Brownian Motion, Martingales, and Stochastic Calculus[J]. 2018.
%	\bibitem{ref2}Morimoto H. Stochastic control and mathematical modeling[M]. Cambridge University Press, 2010.
%	\bibitem{ref3}Athreya K B, Lahiri S N. Measure theory and probability theory[M]. Springer Science \& Business Media, 2006.  
%	\bibitem{ref4}Pivato M. Stoshastic processes and stochastic integration[J]. 1999.
%\end{thebibliography}

\end{document}
